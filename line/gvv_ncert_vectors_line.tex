\let\negmedspace\undefined
\let\negthickspace\undefined
%\RequirePackage{amsmath}
\documentclass[journal,12pt,twocolumn]{IEEEtran}
%
% \usepackage{setspace}
% \usepackage{gensymb}
%\doublespacing
%\singlespacing
%\usepackage{silence}
%Disable all warnings issued by latex starting with "You have..."
%\usepackage{graphicx}
%\usepackage{amssymb}
%\usepackage{relsize}
\usepackage[cmex10]{amsmath}
%\usepackage{amsthm}
%\interdisplaylinepenalty=2500
%\savesymbol{iint}
%\usepackage{txfonts}
%\restoresymbol{TXF}{iint}
%\usepackage{wasysym}
\usepackage{amsthm}
%\usepackage{iithtlc}
% \usepackage{mathrsfs}
% \usepackage{txfonts}
% \usepackage{stfloats}
% \usepackage{steinmetz}
% \usepackage{bm}
% \usepackage{cite}
% \usepackage{cases}
% \usepackage{subfig}
%\usepackage{xtab}
\usepackage{longtable}
%\usepackage{multirow}
%\usepackage{algorithm}
%\usepackage{algpseudocode}
\usepackage{enumitem}
 \usepackage{mathtools}
% \usepackage{tikz}
% \usepackage{circuitikz}
% \usepackage{verbatim}
%\usepackage{tfrupee}
\usepackage[breaklinks=true]{hyperref}
%\usepackage{stmaryrd}
%\usepackage{tkz-euclide} % loads  TikZ and tkz-base
%\usetkzobj{all}
\usepackage{listings}
    \usepackage{color}                                            %%
    \usepackage{array}                                            %%
    \usepackage{longtable}                                        %%
    \usepackage{calc}                                             %%
    \usepackage{multirow}                                         %%
    \usepackage{hhline}                                           %%
    \usepackage{ifthen}                                           %%
  %optionally (for landscape tables embedded in another document): %%
    \usepackage{lscape}     
% \usepackage{multicol}
% \usepackage{chngcntr}
%\usepackage{enumerate}

%\usepackage{wasysym}
%\newcounter{MYtempeqncnt}
\DeclareMathOperator*{\Res}{Res}
%\renewcommand{\baselinestretch}{2}
\renewcommand\thesection{\arabic{section}}
\renewcommand\thesubsection{\thesection.\arabic{subsection}}
\renewcommand\thesubsubsection{\thesubsection.\arabic{subsubsection}}

\renewcommand\thesectiondis{\arabic{section}}
\renewcommand\thesubsectiondis{\thesectiondis.\arabic{subsection}}
\renewcommand\thesubsubsectiondis{\thesubsectiondis.\arabic{subsubsection}}

% correct bad hyphenation here
\hyphenation{op-tical net-works semi-conduc-tor}
\def\inputGnumericTable{}                                 %%

\lstset{
%language=C,
frame=single, 
breaklines=true,
columns=fullflexible
}
%\lstset{
%language=tex,
%frame=single, 
%breaklines=true
%}

\begin{document}
%


\newtheorem{theorem}{Theorem}[section]
\newtheorem{problem}{Problem}
\newtheorem{proposition}{Proposition}[section]
\newtheorem{lemma}{Lemma}[section]
\newtheorem{corollary}[theorem]{Corollary}
\newtheorem{example}{Example}[section]
\newtheorem{definition}[problem]{Definition}
%\newtheorem{thm}{Theorem}[section] 
%\newtheorem{defn}[thm]{Definition}
%\newtheorem{algorithm}{Algorithm}[section]
%\newtheorem{cor}{Corollary}
\newcommand{\BEQA}{\begin{eqnarray}}
\newcommand{\EEQA}{\end{eqnarray}}
\newcommand{\define}{\stackrel{\triangle}{=}}

\bibliographystyle{IEEEtran}
%\bibliographystyle{ieeetr}


\providecommand{\mbf}{\mathbf}
\providecommand{\pr}[1]{\ensuremath{\Pr\left(#1\right)}}
\providecommand{\qfunc}[1]{\ensuremath{Q\left(#1\right)}}
\providecommand{\sbrak}[1]{\ensuremath{{}\left[#1\right]}}
\providecommand{\lsbrak}[1]{\ensuremath{{}\left[#1\right.}}
\providecommand{\rsbrak}[1]{\ensuremath{{}\left.#1\right]}}
\providecommand{\brak}[1]{\ensuremath{\left(#1\right)}}
\providecommand{\lbrak}[1]{\ensuremath{\left(#1\right.}}
\providecommand{\rbrak}[1]{\ensuremath{\left.#1\right)}}
\providecommand{\cbrak}[1]{\ensuremath{\left\{#1\right\}}}
\providecommand{\lcbrak}[1]{\ensuremath{\left\{#1\right.}}
\providecommand{\rcbrak}[1]{\ensuremath{\left.#1\right\}}}
\theoremstyle{remark}
\newtheorem{rem}{Remark}
\newcommand{\sgn}{\mathop{\mathrm{sgn}}}
\providecommand{\abs}[1]{\left\vert#1\right\vert}
\providecommand{\res}[1]{\Res\displaylimits_{#1}} 
\providecommand{\norm}[1]{\left\lVert#1\right\rVert}
%\providecommand{\norm}[1]{\lVert#1\rVert}
\providecommand{\mtx}[1]{\mathbf{#1}}
\providecommand{\mean}[1]{E\left[ #1 \right]}
\providecommand{\fourier}{\overset{\mathcal{F}}{ \rightleftharpoons}}
%\providecommand{\hilbert}{\overset{\mathcal{H}}{ \rightleftharpoons}}
\providecommand{\system}{\overset{\mathcal{H}}{ \longleftrightarrow}}
	%\newcommand{\solution}[2]{\textbf{Solution:}{#1}}
\newcommand{\solution}{\noindent \textbf{Solution: }}
\newcommand{\cosec}{\,\text{cosec}\,}
\providecommand{\dec}[2]{\ensuremath{\overset{#1}{\underset{#2}{\gtrless}}}}
\newcommand{\myvec}[1]{\ensuremath{\begin{pmatrix}#1\end{pmatrix}}}
\newcommand{\mydet}[1]{\ensuremath{\begin{vmatrix}#1\end{vmatrix}}}
%\numberwithin{equation}{section}
\numberwithin{equation}{subsection}
%\numberwithin{problem}{section}
%\numberwithin{definition}{section}
\makeatletter
\@addtoreset{figure}{problem}
\makeatother

\let\StandardTheFigure\thefigure
\let\vec\mathbf
%\renewcommand{\thefigure}{\theproblem.\arabic{figure}}
\renewcommand{\thefigure}{\theproblem}
%\setlist[enumerate,1]{before=\renewcommand\theequation{\theenumi.\arabic{equation}}
%\counterwithin{equation}{enumi}


%\renewcommand{\theequation}{\arabic{subsection}.\arabic{equation}}

\def\putbox#1#2#3{\makebox[0in][l]{\makebox[#1][l]{}\raisebox{\baselineskip}[0in][0in]{\raisebox{#2}[0in][0in]{#3}}}}
     \def\rightbox#1{\makebox[0in][r]{#1}}
     \def\centbox#1{\makebox[0in]{#1}}
     \def\topbox#1{\raisebox{-\baselineskip}[0in][0in]{#1}}
     \def\midbox#1{\raisebox{-0.5\baselineskip}[0in][0in]{#1}}

\vspace{3cm}

\title{
	%\logo{
%Computational Approach to School Geometry
Points and Vectors
%	}
}
\author{ G V V Sharma$^{*}$% <-this % stops a space
	\thanks{*The author is with the Department
		of Electrical Engineering, Indian Institute of Technology, Hyderabad
		502285 India e-mail:  gadepall@iith.ac.in. All content in this manual is released under GNU GPL.  Free and open source.}
	
}	
%\title{
%	\logo{Matrix Analysis through Octave}{\begin{center}\includegraphics[scale=.24]{tlc}\end{center}}{}{HAMDSP}
%}


% paper title
% can use linebreaks \\ within to get better formatting as desired
%\title{Matrix Analysis through Octave}
%
%
% author names and IEEE memberships
% note positions of commas and nonbreaking spaces ( ~ ) LaTeX will not break
% a structure at a ~ so this keeps an author's name from being broken across
% two lines.
% use \thanks{} to gain access to the first footnote area
% a separate \thanks must be used for each paragraph as LaTeX2e's \thanks
% was not built to handle multiple paragraphs
%

%\author{<-this % stops a space
%\thanks{}}
%}
% note the % following the last \IEEEmembership and also \thanks - 
% these prevent an unwanted space from occurring between the last author name
% and the end of the author line. i.e., if you had this:
% 
% \author{....lastname \thanks{...} \thanks{...} }
%                     ^------------^------------^----Do not want these spaces!
%
% a space would be appended to the last name and could cause every name on that
% line to be shifted left slightly. This is one of those "LaTeX things". For
% instance, "\textbf{A} \textbf{B}" will typeset as "A B" not "AB". To get
% "AB" then you have to do: "\textbf{A}\textbf{B}"
% \thanks is no different in this regard, so shield the last } of each \thanks
% that ends a line with a % and do not let a space in before the next \thanks.
% Spaces after \IEEEmembership other than the last one are OK (and needed) as
% you are supposed to have spaces between the names. For what it is worth,
% this is a minor point as most people would not even notice if the said evil
% space somehow managed to creep in.

%\WarningFilter{latex}{LaTeX Warning: You have requested, on input line 117, version}


% The paper headers
%\markboth{Journal of \LaTeX\ Class Files,~Vol.~6, No.~1, January~2007}%
%{Shell \MakeLowercase{\textit{et al.}}: Bare Demo of IEEEtran.cls for Journals}
% The only time the second header will appear is for the odd numbered pages
% after the title page when using the twoside option.
% 
% *** Note that you probably will NOT want to include the author's ***
% *** name in the headers of peer review papers.                   ***
% You can use \ifCLASSOPTIONpeerreview for conditional compilation here if
% you desire.




% If you want to put a publisher's ID mark on the page you can do it like
% this:
%\IEEEpubid{0000--0000/00\$00.00~\copyright~2007 IEEE}
% Remember, if you use this you must call \IEEEpubidadjcol in the second
% column for its text to clear the IEEEpubid mark.



% make the title area
\maketitle

\newpage

\tableofcontents

\bigskip

\renewcommand{\thefigure}{\theenumi}
\renewcommand{\thetable}{\theenumi}
%\renewcommand{\theequation}{\theenumi}

%\begin{abstract}
%%\boldmath
%In this letter, an algorithm for evaluating the exact analytical bit error rate  (BER)  for the piecewise linear (PL) combiner for  multiple relays is presented. Previous results were available only for upto three relays. The algorithm is unique in the sense that  the actual mathematical expressions, that are prohibitively large, need not be explicitly obtained. The diversity gain due to multiple relays is shown through plots of the analytical BER, well supported by simulations. 
%
%\end{abstract}
% IEEEtran.cls defaults to using nonbold math in the Abstract.
% This preserves the distinction between vectors and scalars. However,
% if the journal you are submitting to favors bold math in the abstract,
% then you can use LaTeX's standard command \boldmath at the very start
% of the abstract to achieve this. Many IEEE journals frown on math
% in the abstract anyway.

% Note that keywords are not normally used for peerreview papers.
%\begin{IEEEkeywords}
%Cooperative diversity, decode and forward, piecewise linear
%\end{IEEEkeywords}



% For peer review papers, you can put extra information on the cover
% page as needed:
% \ifCLASSOPTIONpeerreview
% \begin{center} \bfseries EDICS Category: 3-BBND \end{center}
% \fi
%
% For peerreview papers, this IEEEtran command inserts a page break and
% creates the second title. It will be ignored for other modes.
%\IEEEpeerreviewmaketitle

\begin{abstract}
This book provides a computational approach to school geometry based on the NCERT textbooks from Class 6-12.  Links to sample Python codes are available in the text.  
\end{abstract}

\section{Norm and Direction Vector}
\renewcommand{\theequation}{\theenumi}
%\begin{enumerate}[label=\arabic*.,ref=\theenumi]
\begin{enumerate}[label=\thesubsection.\arabic*.,ref=\thesubsection.\theenumi]
\numberwithin{equation}{enumi}
\item Find the equation of set of points $\vec{P}$ such that
\begin{align}
PA^2+PB^2 =2k^2,
\end{align}
%
\begin{align}
\vec{A} = \myvec{3\\4 \\5},
\vec{B} = \myvec{-1\\3 \\-7},
\end{align}
%
respectively.
%
%
\solution
%\input{solutions/su2021/2/25.tex}   

\item Find a condition on $\vec{x}$  such that the points $\vec{x}, \myvec{1\\2}, \myvec{7\\0}$ are collinear.
\\
\solution
%\input{solutions/aug/2/11.tex}
\item Find the direction vectors of the sides of a triangle with vertices
$
\vec{A} = \myvec{3\\5 \\-4},
\vec{B} = \myvec{-1\\1 \\2}, \text{ and }
\vec{C} = \myvec{-5\\ -5\\-2}
$
\\
\solution
%\input{solutions/aug/2/4.tex}
\item Find a unit vector in the direction of 
%
\begin{align}
\myvec{1\\1\\-2}.
\end{align}
%
\solution
%\input{solutions/aug/2/24.tex}

%
\item Find a unit vector in the direction of \myvec{2\\-1\\-2}.
\\
\solution
%\input{solutions/aug/2/21.tex}

%
%



\item Find the equation of the set of points $\vec{P}$ such that its distances from the points
$
\vec{A}=\myvec{3\\4\\-5}, 
\vec{B}=\myvec{-2\\1\\4}
$
are equal. 
\\
\solution
%\input{solutions/su2021/2/29/main.tex}
\item Find a unit vector in the direction of the line passing through \myvec{-2\\4\\-5} and $\myvec{1\\2\\3}$.
%
\\
\solution
%\input{solutions/aug/2/22.tex}
\item Find a point on the $y$-axis which is equidistant from the points $\vec{A} = \myvec{6\\5}, \vec{B} = \myvec{-4\\3}$.
\\
\solution
%\input{solutions/july/25/main1.tex}
\end{enumerate}
\section{Inner}
\renewcommand{\theequation}{\theenumi}
%\begin{enumerate}[label=\arabic*.,ref=\theenumi]
\begin{enumerate}[label=\thesection.\arabic*.,ref=\thesection.\theenumi]
\numberwithin{equation}{enumi}
\item The line through the points \myvec{-2\\6} and \myvec{4\\8} is perpendicular to the line through the points \myvec{8\\12} and $\myvec{x\\24}$.  Find the value of $x$.
\\
\solution
%\input{solutions/su2021/2/36/main.tex}
\item Show that the line joining the origin to the point \myvec{2\\1\\1} is perpendicular to the line determined by the points \myvec{3\\5\\-1}, \myvec{4\\3\\-1}.
\\
\solution
%\input{solutions/su2021/2/37.tex}
\item Are the points 
\begin{align}
\vec{A} = \myvec{3\\6 \\9},
\vec{B} = \myvec{10\\20 \\30},
\vec{C} = \myvec{25\\ -41\\5},
\end{align}
%
the vertices of a right angled triangle?
\\
\solution
%\input{solutions/july/2/6/Assignment4.tex}
\item Show that the vectors 
$
\myvec{2\\-1 \\1},
\myvec{1\\-3 \\-5},
\myvec{3\\ -4\\-4}
$
form the vertices of a right angled triangle.
\\
\solution
%\input{solutions/aug/2/3/Assignment-1.tex}
\item Show that the points 
\begin{align}
\vec{A} = \myvec{2\\-1 \\1},
\vec{B} = \myvec{1\\-3 \\-5},
\vec{C} = \myvec{3\\ -4\\-4}
\end{align}
%
are the vertices of a right angled triangle.
\\
\solution
%\input{solutions/su2021/2/4/Assignment 8 (3).tex}


\item In $\triangle ABC$, 
$
\vec{A} = \myvec{1\\2 \\3},
\vec{B} = \myvec{-1\\0 \\0},
\vec{C} = \myvec{0\\ 1\\2}.
$
Find $\angle B$.
\\
\solution
%\input{solutions/su2021/2/5/main.tex}
\item Without using the Pythagoras theorem, show that the points \myvec{4\\ 4}, \myvec{3\\ 5} and \myvec{–1\\ –1} are the vertices of a right angled triangle.
\\
\solution
%\input{./solutions/triangle/18/solution.tex}
\item Show that the points 
\begin{align}
\vec{A} = \myvec{2\\-1 \\1},
\vec{B} = \myvec{1\\-3 \\-5},
\vec{C} = \myvec{3\\ -4\\-4}
\end{align}
%
are the vertices of a right angled triangle.
\\
\solution 
The following code plots Fig. \ref{fig:triangle_3d}
%
\begin{lstlisting}
codes/triangle/triangle_3d.py
\end{lstlisting}
%
%\begin{figure}[!ht]
%\includegraphics[width=\columnwidth]{./triangle/figs/triangle_3d.eps}
%\caption{}
%\label{fig:triangle_3d}
%\end{figure}
%
From the figure, it appears that $\triangle ABC$ is right angled at $\vec{C}$.  Since 
\begin{align}
\brak{\vec{A}-\vec{C}}^T\brak{\vec{B}-\vec{C}}&=0
\end{align}
%
it is proved that the triangle is indeed right angled.
\end{enumerate}
\section{Rank}
\renewcommand{\theequation}{\theenumi}
%\begin{enumerate}[label=\arabic*.,ref=\theenumi]
\begin{enumerate}[label=\thesection.\arabic*.,ref=\thesection.\theenumi]
\numberwithin{equation}{enumi}
\item Prove that the three points \myvec{-4\\6\\10}, \myvec{2\\4\\6} and \myvec{14\\0\\-2} are collinear.
%
\\
\solution 
%
%\input{solutions/su2021/2/27/main.tex}
\item Show that 
$
\vec{A}=\myvec{2\\3\\-4}, 
\vec{B}=\myvec{1\\-2\\3} \text{ and } 
\vec{C}=\myvec{3\\8\\-11}$  
are collinear.
\\
\solution 
%
%\input{solutions/su2021/2/24/main.tex}
\item Do the points $\vec{A}=\myvec{3\\2}, \vec{B}=\myvec{-2\\-3}, \vec{C}=\myvec{2\\3} $ form a triangle?  If so, name the type of triangle formed.
\label{prob:tri_exam_coll_pts}
%
\\
\solution 

The direction vectors of $AB$ and $BC$ are 
\begin{align}
\label{eq:tri_geo_ex_baorth}
\vec{B}-\vec{A} &= \myvec{-5\\-5}
\\
\vec{C}-\vec{A} &= \myvec{-1\\1}
\label{eq:tri_geo_ex_caorth}
\end{align}
%
If $\vec{A}, \vec{B}, \vec{C}$ form a line, then, $AB$ and $AC$ should have the same direction vector. Hence, there exists a $k$ such that
\begin{align}
\vec{B}-\vec{A} &= k\brak{\vec{C}-\vec{B}}
\\
\implies \vec{B} &= \frac{k\vec{C} +\vec{A}}{k+1}
\label{eq:tri_geo_ex_caorth_section}
\end{align}
%
Since 
\begin{align}
\vec{B}-\vec{A} \ne k\brak{\vec{C}-\vec{A}},
\end{align}
%
the points are not collinear and form a triangle.  An alternative method is to create the matrix
\begin{align}
\label{eq:tri_geo_ex_diff_mat}
\vec{M} = \myvec{\vec{B}-\vec{A} & \vec{B}-\vec{A}}^T 
\end{align}
%
If $rank(\vec{M}) = 1$, the points are collinear.  The rank of a matrix is the number of nonzero rows left after doing row operations.  In this problem, 
%
\begin{align}
\vec{M} = \myvec{-5 & -5\\-1 & 1}\xleftrightarrow {R_2\leftarrow 5R_2-R_1}\myvec{-5 & -5\\0 & 10}
\\
\implies rank(\vec{M}) = 2
\end{align}
%
as the number of non zero rows is 2.
The following code plots Fig. \ref{fig:check_tri}
%
\begin{lstlisting}
codes/triangle/check_tri.py
\end{lstlisting}
%
%\begin{figure}[!ht]
%\includegraphics[width=\columnwidth]{./triangle/figs/check_tri.eps}
%\caption{}
%\label{fig:check_tri}
%\end{figure}
%
From the figure, it appears that $\triangle ABC$ is right angled, with $BC$ as the hypotenuse.  From Baudhayana's theorem, this would be true if 
\begin{align}
\norm{\vec{B}-\vec{A}}^2+\norm{\vec{C}-\vec{A}}^2&=\norm{\vec{B}-\vec{C}}^2
\end{align}
which can be expressed as
\begin{multline}
\norm{\vec{A}}^2 + \norm{\vec{C}}^2 - 2\vec{A}^T\vec{C}+
\norm{\vec{A}}^2 + \norm{\vec{B}}^2 - 2\vec{A}^T\vec{B}
\\
=
\norm{\vec{B}}^2 + \norm{\vec{C}}^2 - 2\vec{B}^T\vec{C}
\end{multline}
%
to obtain 
\begin{align}
\label{eq:tri_geo_ex_orth}
\brak{\vec{B}-\vec{A}}^T\brak{\vec{C}-\vec{A}}&=0
\end{align}
%
after simplification.  From \eqref{eq:tri_geo_ex_baorth} and \eqref{eq:tri_geo_ex_caorth}, it is easy to verify that 
\begin{align}
\label{eq:tri_geo_ex_orth_sol}
\brak{\vec{B}-\vec{A}}^T\brak{\vec{C}-\vec{A}}=
 \myvec{-5 & -5}\myvec{-1\\1} = 0
\end{align}
satisfying
\eqref{eq:tri_geo_ex_orth}. Thus,  $\triangle ABC$ is right angled at $\vec{A}$.
%
\end{enumerate}
\section{Cross}
\renewcommand{\theequation}{\theenumi}
%\begin{enumerate}[label=\arabic*.,ref=\theenumi]
\begin{enumerate}[label=\thesection.\arabic*.,ref=\thesection.\theenumi]
\numberwithin{equation}{enumi}
%\renewcommand{\theequation}{\theenumi}
%\begin{enumerate}[label=\thesubsection.\arabic*.,ref=\thesubsection.\theenumi]
%\numberwithin{equation}{enumi}
%
%
\item Find the area of a triangle having the points 
$
\vec{A} = \myvec{1\\1 \\1},
\vec{B} = \myvec{1\\2 \\3}, \text{ and }
\vec{C} = \myvec{2\\ 3\\1}
$
as its vertices.
\\
\solution
%\input{solutions/su2021/2/7.tex}
%
\item Find the area of a triangle with vertices
$
\vec{A} = \myvec{1\\1 \\2},
\vec{B} = \myvec{2\\3 \\5}, \text{ and }
\vec{C} = \myvec{1\\ 5\\5}
$
\\
\solution
%\input{solutions/su2021/2/8.tex}
%
\item Find the area of the triangle whose vertices are
\begin{enumerate}
\item \myvec{2\\3}, \myvec{-1\\0},  \myvec{2\\-4}
\item  \myvec{-5\\-1},  \myvec{3\\-5},  \myvec{5\\2}
\end{enumerate}
\solution
%\input{./solutions/6/chapters/triangle/solution.tex}

\item Find the area of the triangle formed by joining the mid points of the sides of a triangle whose vertices are  \myvec{0\\-1},  \myvec{2\\1},  \myvec{0\\3}.
\\
\solution
%\input{./solutions/7/chapters/triangle/solution.tex}
\item Verify that the median of $\triangle ABC$ with vertices $\vec{A}=\myvec{4\\-6},  \vec{B}=\myvec{3\\-2}$ and  $\vec{C} =  \myvec{5\\2}$ divides it into two triangles of equal areas.
\\
\solution
%\input{./solutions/8/chapters/solution.tex}
\item Find the area of a triangle whose vertices are 
$\vec{A}=\myvec{1\\-1}, 
\vec{B} = \myvec{-4\\6}$ and
$ 
\vec{C} = \myvec{-3\\-5}
$.
%
\\
\solution
  Using Hero's formula, the following code computes the area of the  triangle as 24.
%
\begin{lstlisting}
codes/triangle/area_tri.py
\end{lstlisting}
%
%
\item Find the area of a triangle formed by the vertices $\vec{A}=\myvec{5\\2}, \vec{B}=\myvec{4\\7}, \vec{C}=\myvec{7\\-4}$.
%\\
\solution  The area of $\triangle ABC$ is also obtained  in terms of the  {\em magnitude} of the determinant of the matrix $\vec{M}$ in  \eqref{eq:tri_geo_ex_diff_mat} as
%
\begin{align}
\frac{1}{2}\mydet{\vec{M}}
\end{align}
The computation is done in \textbf{area\_tri.py}
\item Find the area of a triangle formed by the points $\vec{P}=\myvec{-1.5\\3}, \vec{Q}=\myvec{6\\-2}, \vec{R}=\myvec{-3\\4}$.
\\
\solution Another formula for the area of $\triangle ABC$  is
%
\begin{align}
\frac{1}{2}\mydet{1 & 1 & 1\\ \vec{A} & \vec{B} & \vec{C} }
\end{align}
%
\item Find the area of a triangle having the points
%
\begin{align}
\vec{A} = \myvec{1\\1 \\1},
\vec{B} = \myvec{1\\2 \\3},
\vec{C} = \myvec{2\\ 3\\1}
\end{align}
%
as its vertices.
\\
\solution The area of a triangle using the {\em vector product} is obtained as
\begin{align}
\frac{1}{2}\norm{\brak{\vec{B}-\vec{A}}\times \brak{\vec{C}-\vec{A}}}
\end{align}
%
For any two vectors $\vec{a}=\myvec{a_1\\a_2\\a_3}, \vec{b}=\myvec{b_1\\b_2\\b_3}$, 
\begin{align}
\label{eq:tri_cross_prod}
\vec{a}\times \vec{b} = \myvec{0 & -a_3 & a_2 \\ a_3 & 0 & -a_1 \\ -a_2 & a_1 & 0}\myvec{b_1\\b_2\\b_3}
\end{align}
%
The following code computes the area using the vector product.
%
\begin{lstlisting}
codes/triangle/area_tri_vec.py
\end{lstlisting}
%
\end{enumerate}
\section{Section}
\renewcommand{\theequation}{\theenumi}
%\begin{enumerate}[label=\arabic*.,ref=\theenumi]
\begin{enumerate}[label=\thesection.\arabic*.,ref=\thesection.\theenumi]
\numberwithin{equation}{enumi}
%\renewcommand{\theequation}{\theenumi}
%\begin{enumerate}[label=\thesubsection.\arabic*.,ref=\thesubsection.\theenumi]
%\numberwithin{equation}{enumi}
%
\item Find the coordinates of a point which divides the line segment joining the points \myvec{1\\-2\\3} and \myvec{3\\4\\-5} in the ratio $2:3$
\begin{enumerate}
\item internally, and
\item externally.
\end{enumerate}
%
\solution
%\input{solutions/su2021/2/26/main.tex}
\item Let $\vec{A}=\myvec{4\\2},  \vec{B}=\myvec{6\\5}$ and  $\vec{C} =  \myvec{1\\4}$ be the vertices of $\triangle ABC$.
\begin{enumerate}
\item The median from $\vec{A}$ meets $BC$ at $\vec{D}$.  Find the coordinates of the point $\vec{D}$.
\item Find the coordinates of the point $\vec{P}$ on $AD$ such that $AP:PD = 2:1$.
\item Find the coordinates of the points $\vec{Q}$ and $\vec{R}$ on medians $BE$ and $CF$ respectively such that $BQ:QE = 2:1$ and $CR:RF = 2:1$.
\end{enumerate}
\solution
%\input{solutions/su2021/2/2/Assignment-9.tex}
%
\item The centroid of a $\triangle ABC$ is at the point \myvec{1\\1\\1}.  If the coordinates of $\vec{A}$ and $\vec{B}$ are \myvec{3\\-5\\7} and \myvec{-1\\7\\-6}, respectively, find the coordinates of the point $\vec{C}$.
%
\\
\solution The centroid of $\triangle ABC$ is given by
\begin{align}
\label{eq:tri_geo_ex_centroid}
\vec{O} = \frac{\vec{A}+\vec{B}+\vec{C}}{3}
\end{align}
%
Thus, 
\begin{align}
\vec{C} = 3\vec{C}-\vec{A}-\vec{B}
\end{align}
%
\end{enumerate}

\section{Linear Equations}
\renewcommand{\theequation}{\theenumi}
%\begin{enumerate}[label=\arabic*.,ref=\theenumi]
\begin{enumerate}[label=\thesection.\arabic*.,ref=\thesection.\theenumi]
\numberwithin{equation}{enumi}
\item Draw the graphs of the equations 
\begin{align}
\label{eq:1.2.1_p1}
\myvec{1 & -1}\vec{x} + 1 &= 0 
\\
\myvec{ 3 & 2}\vec{x} - 12 &= 0
\label{eq:1.2.1_p2}
\end{align}
%
  Determine the coordinates of the vertices of the triangle formed by these lines and the x-axis, and shade the triangular region.
\\
\solution
%\input{./solutions/1/chapters/triangle/solution.tex}
%
\item In a $\triangle ABC, \angle C = 3 \angle B = 2 (\angle A + \angle B)$. Find the three angles. 
\\
\solution
%\input{./solutions/2/chapters/triangle_ex/solution.tex}
\item Draw the graphs of the equations $5x – y = 5$ and $3x – y = 3$. Determine the co-ordinates of the vertices of the triangle formed by these lines and the y axis.
\\
\solution
%\input{./solutions/3/chapters/triangle/solution.tex}

\end{enumerate}
\section{Linear Forms}
\renewcommand{\theequation}{\theenumi}
%\begin{enumerate}[label=\arabic*.,ref=\theenumi]
\begin{enumerate}[label=\thesection.\arabic*.,ref=\thesection.\theenumi]
\numberwithin{equation}{enumi}
\item The vertices of $\triangle PQR$ are 

$
\vec{P} = \myvec{2 \\1},
\vec{Q} = \myvec{-2\\3},
\vec{R} = \myvec{4\\5}.
$
Find the equation of the median through the vertex $\vec{R}$.
\\
\solution
%\input{./solutions/4/chapters/triangle/solution.tex}
\item In the $\triangle ABC$ with vertices
$
\vec{A}=\myvec{2\\3}, 
\vec{B}=\myvec{4\\-1},
 \vec{C}=\myvec{1\\2}
$,
find the equation and length of the altitude from the vertex $\vec{A}$.
\\
\solution
%\input{./solutions/5/chapters/triangle/docq2.tex}
%\solution Use \eqref{eq:line_section_form}.

\end{enumerate}
\end{document}


