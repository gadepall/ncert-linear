
\section{Elementary}
 \item Verify whether the following are zeroes of the polynomial, indicated against them. 
\begin{enumerate}
\item $ p(x) = 3x + 1, x = \frac{1}{3}$
\item $ p(x) = 5x -\pi, x = \frac{4}{5}$
\item $ p(x) = 5lx+m, x = -\frac{m}{l}$
\item $ p(x) = 2x+1, x = \frac{1}{2}$
\end{enumerate}
%
\solution 
\input{./solutions/1/chapters/line/line_plane/solution.tex}
%
\item Find the zero of the polynomial in each of the following cases: 
\begin{enumerate}
\item $p(x) = x + 5 $
\item $p(x) = x – 5$
\item $p(x) = 2x + 5$
\item $p(x) = 3x – 2$
 \item $p(x) = 3x$
 \item $p(x) = ax, a \ne 0$
\item $p(x) = cx + d, c \ne 0, c, d$ are real numbers.
\end{enumerate}
\solution 
\input{./solutions/2/chapters/line_ex/lines_and_planes/solution.tex}
\item Find two solutions for each of the following equations: 
\begin{enumerate}
\item $\myvec{4 & 3}\vec{x} = 12$
\item $\myvec{ 2 & 5}\vec{x}  = 0 $
\item $\myvec{ 0 & 3}\vec{x}  = 4$
\end{enumerate}
\solution 
\input{./solutions/3/chapters/line/pointonline/solution.tex}

\item Sketch the following lines
%
\begin{enumerate}[itemsep=2pt]
%\begin{multicols}{2}
\item
$
\myvec{2 & 3 }\vec{x}=9.35
$
\item
$
\myvec{1 & -\frac{1}{5} }\vec{x}=10
$
\item
$
\myvec{-2 & 3 }\vec{x}=6
$
\item
$
\myvec{1 & -3 }\vec{x}=0
$
\item
$
\myvec{2 & 5 }\vec{x}=0
$
\item
$
\myvec{3 & 0 }\vec{x}=-2
$
\item
$
\myvec{0 & 1 }\vec{x}=2
$
\item
$
\myvec{2 & 0 }\vec{x}=5
$
%\end{multicols}
\end{enumerate}
%
\solution 
\input{./solutions/4/chapters/line/lines_and_plane/solution.tex}
%
\item Draw the graphs of the following equations
\begin{enumerate}[itemsep=2pt]
\begin{multicols}{2}
\item $\myvec{1 & 1}\vec{x} = 0$
\item $\myvec{ 2 & -1}\vec{x}  = 0 $
\item $\myvec{ 1 & -1}\vec{x}  = 0$
\item $\myvec{ 2 & -1}\vec{x}  = -1$
\item $\myvec{ 2 & -1}\vec{x}  = 4$
\item $\myvec{ 1& -1}\vec{x}  = 4$
\end{multicols}
\end{enumerate}
\solution 
\input{./solutions/5/chapters/lines/docq11.tex}
%
\item Write four solutions for each of the following equations
\begin{enumerate}
\item $\myvec{2 & 1}\vec{x} = 7$
\item $\myvec{ \pi & 1}\vec{x}  = 9 $
\item $\myvec{ 1 & -4}\vec{x}  = 0$
\end{enumerate}
\solution 
\input{./solutions/6/chapters/line/lines_planes/solution.tex}
\item Find the intercepts of the following  lines on the axes.
\begin{enumerate}
\item $\myvec{3 & 2}\vec{x} = 12$.
\item $\myvec{4 & -3}\vec{x} = 6$.
\item $\myvec{3 & 2}\vec{x} = 0$.
\end{enumerate}
\solution
\input{./solutions/line_plane/30/solution.tex}
%
\section{Elementary}
\item If the point \myvec{3 \\ 4} lies on the graph of the equation $3y = ax + 7$, find the value of $a$
\\
\solution
\input{solutions/su2021/2/5/Assignment4.tex}
\item Given the linear equation $\myvec{2 & 3}\vec{x} – 8 = 0$, write another linear equation in two variables such that the geometrical representation of the pair so formed is: 
%
\begin{enumerate}[itemsep=2pt]
\begin{multicols}{2}
\item  intersecting lines
\item parallel lines 
\item  coincident lines
\end{multicols}
\end{enumerate}
%
\solution
\input{solutions/su2021/2/9/main.tex}

\item Check whether –2 and 2 are zeroes of the polynomial $x + 2.$
\\
\solution Let 
%
\begin{align}
y &= x + 2
\implies \myvec{-1 & 1}\vec{x} &= 2
\end{align}
%
Thus, 
%
\begin{align}
y &= 0 
\\
\implies  x + 2 &=0
\\
\text{or, } x &= -2
\end{align}
%
Hence -2 is a zero. This is verified in Fig. \ref{fig:line_zeros}.
%
\begin{figure}[!ht]
\includegraphics[width=\columnwidth]{./line/figs/line_zeros.eps}
\caption{}
\label{fig:line_zeros}
\end{figure}
%
\item Find a zero of the polynomial $p(x) = 2x + 1$.
\\
\solution $p\brak{-\frac{1}{2}} = 0$.
%
%
\item Find four different solutions of the equation 
\label{prob:line_icept}
%
\begin{align}
\label{eq:line_iceptx}
\myvec{1 & 2}\vec{x} &= 6
\end{align}
%
\solution Let 
%
\begin{align}
\vec{x} = \myvec{a\\0}
\end{align}
%
Substituting in \eqref{eq:line_iceptx}, 
%
\begin{align}
\myvec{1 & 2} \myvec{a\\0}&= 6
\\
\implies a &=6
\end{align}
%
Simiarly, substituting 
%
\begin{align}
\vec{x} &= \myvec{0\\b},
\end{align}
%
in \eqref{eq:line_iceptx}, 
%
%
\begin{align}
b =3
\end{align}
%
More solutions can be obtained in a similar fashion.
%
\item Draw the graph of 
%
\begin{align}
\myvec{1 & 1}\vec{x} &= 7
\end{align}
%
\solution The intercepts on the x and y-axis can be obtained from Problem \ref{prob:line_icept}
as
%
\begin{align}
\vec{A} = \myvec{7\\0}
\vec{B} = \myvec{0\\7}
\end{align}
%
The following python code can be used to draw the graph in Fig. \ref{fig:line_icept}.
%
\begin{lstlisting}
codes/line/line_icept.py
\end{lstlisting}
%
\begin{figure}[!ht]
\includegraphics[width=\columnwidth]{./line/figs/line_icept.eps}
\caption{}
\label{fig:line_icept}
\end{figure}
%
%
%

%
%
%
%
%
